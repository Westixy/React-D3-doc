\documentclass[a4paper, french, 12pt]{extarticle}
  \usepackage{datetime}
  \usepackage{babel}
  \usepackage[utf8]{inputenc}
  \usepackage[T1]{fontenc}
  \usepackage{lmodern}
  \usepackage{listings}
  \usepackage{color}
  \usepackage{blindtext}
  \usepackage{scrextend}
  
  \addtokomafont{labelinglabel}{\sffamily}
  \definecolor{lightgray}{rgb}{.9,.9,.9}
  \definecolor{darkgray}{rgb}{.4,.4,.4}
  \definecolor{purple}{rgb}{0.65, 0.12, 0.82}
  
  \lstdefinelanguage{JavaScript}{
    keywords={typeof, new, true, false, catch, function, return, null, catch, switch, var, let, const, if, in, while, do, else, case, break},
    keywordstyle=\color{blue}\bfseries,
    ndkeywords={class, export, boolean, throw, implements, import, from, this},
    ndkeywordstyle=\color{darkgray}\bfseries,
    identifierstyle=\color{black},
    sensitive=false,
    comment=[l]{//},
    morecomment=[s]{/*}{*/},
    commentstyle=\color{purple}\ttfamily,
    stringstyle=\color{red}\ttfamily,
    morestring=[b]',
    morestring=[b]"
  }
  
  \lstset{
     language=JavaScript,
     backgroundcolor=\color{lightgray},
     extendedchars=true,
     basicstyle=\footnotesize\ttfamily,
     showstringspaces=false,
     showspaces=false,
     numbers=left,
     numberstyle=\footnotesize,
     numbersep=9pt,
     tabsize=2,
     breaklines=true,
     showtabs=false,
     captionpos=b
  }
  
  \title{React \& D3}
  \author{Esteban Sotillo\\{\small CPNV-ES}}
  
  \date{{\small \today}}
  % Hint: \title{what ever}, \author{who care} and \date{when ever} could stand 
  % before or after the \begin{document} command 
  % BUT the \maketitle command MUST come AFTER the \begin{document} command! 


\begin{document}
\maketitle
\section{Introduction}
Documentation de l'utilisation de React et D3 afin de pouvoir commencer un projet avec ces technologies de manières simple et rapides.

Pour faire ce projet j'ai travaillé sur une version de linux donc certaines manipulations risque de changer un peu. Les possibles manipulations supplémentaires sont laissées au lecteur.

\section{Mise en place}
\subsection{Prérequis}
\begin{itemize}
  \item npm v5.x+
  \item Syntaxe ES6+
  \item React
\end{itemize}

\subsection{Création de la base du projet}
Pour la base du projet il est plus pratique d'utiliser le CLI de création d'app React afin d'avoir une base de code consistante avec tous les modules principaux.
\begin{lstlisting}[language=bash]
  npm i -g create-react-app
  create-react-app my-app
  cd my-app
\end{lstlisting}

\subsection{Installation des dépendances}
Pour ce projet j'ai utilisé différents packages dont j'expliquerais l'utilité dans les prochaines lignes.

\begin{labeling}{react-chord-diagram}
  \item [react-move] Utilisé pour aider à l'animation de composants react.
  \item [react-chord-diagram] Composant react simplifiant l'utilisation du diagram chord de D3.
  \item [d3-ease] Librairie de fonctions servant à définir le flux des transitions.
\end{labeling}
Pour les installer il suffit d'entrer la commande suivante : 
\begin{lstlisting}[language=bash]
  npm i react-move react-chord-diagram d3-ease
\end{lstlisting}

\subsection{Structure de fichier}
Voici un rapide descriptif des fichiers utiles pour commencer le projet.
\begin{lstlisting}
+ public : fichiers statics modifiable
+ src : dossier de travail
  - index.js : fichier js de base
  - App.js : Composant react originel
\end{lstlisting}

\section{Utilisation}
Dans cette partie, je vais faire un petit projet afin d'expliquer plus facilement l'utilisation des composants.
\subsection{Importation des composants}
Dans le fichier \textbf{src/App.js} ajouter ces trois lignes au début du fichier.
\begin{lstlisting}[language=JavaScript]
  import ChordDiagram from 'react-chord-diagram'
  import { Animate } from 'react-move'
  import { easeExpInOut } from 'd3-ease'
\end{lstlisting}
Puis supprimer le contenu de la première div.
\pagebreak
\subsection{Composant ChordDiagram}
Le composant ChordDiagram possède 4 propriétés que nous allons utiliser qui sont "matrix", "componentId", "groupLabels" et "groupColors".\\
La propriété "matrix" est un tableau à 2 dimensions comprenant en x la quantité partant et en y la quantité de réception.\\
La propriété "componentId" est un ID pour reconnaître le diagramme.\\
La propriété "groupLabels" est un tableau de string utilisé comme label pour le diagramme.\\
La propriété "groupColors" est un tableau de couleurs css utilisé pour le diagramme.\\\\
Exemple : 
\begin{lstlisting}[language=html]
  <ChordDiagram
    matrix={[[1,2],[3,4]]}
    componentId={"ID"}
    groupLabels={["foo", "boo"]}
    groupColors={["#333","#567"]}
  />
\end{lstlisting}


\subsection{Composant Animate}
Le composant Animate sers à aider à l'animation des composants react. Il possède 2 propriétés requises qui sont "start" et "update". Ces deux propriétés sont des références de fonction définissant respectivement les status initiaux et les status après execution.\\\\
Exemple :
\begin{lstlisting}[language=html]
  <Animate
    start={() => ({
      x: 0
    })}
    update={() => ({
      x: [1],
      timing: { duration: 1500, ease: easeExpInOut },
    })}
  >
  {state => null /* do something with state.x */}
  </Animate>
\end{lstlisting}

\subsection{Dans le code}
\subsubsection{Configuration du composant React}
Pour commencer, il faut définir des "state" sur notre composant app (à défaut d'en avoir créer un autre). nous aurons besoin de trois variables différentes qui sont "x", "matrix" et "oldMatrix". Il faut le définir dans le constructeur de notre composant.
\begin{lstlisting}[language=JavaScript]
  constructor() {
    this.state = {
      x: 0,
      matrix: [[]],
      oldMatrix: null,
    }
  }
\end{lstlisting}
Il faut ensuite ajouter un bouton (dans la partie rendu) qui sera utilisé pour modifier les données de notre matrice.
\begin{lstlisting}[language=html]
  <div className="App">
    <input type="button" onClick={()=>{}} value="update" />
  </div>
\end{lstlisting}
Ajouter le composant Animate a la suite du bouton.
\begin{lstlisting}[language=JavaScript]
  <Animate
    start={() => ({
      x: 0,
    })}
    update={() => ({
      x: [this.state.x],
      timing: { duration: 1500, ease: easeExpInOut },
    })}
  >
    {state => {
      let {x} = state
      // la suite se passera ici
    }}
  </Animate>
\end{lstlisting}
\pagebreak
\subsubsection{Calcul des différences entre les matrices}
Afin d'avoir une transition jolie entre la nouvelle et l'ancienne matrice, il faut calculer a l'aide de l'indice x (qui évoluera entre 0 et 1).\\
Le code se trouve dans le contenu du composant Animate. 
\begin{lstlisting}[language=JavaScript]
  let {x} = state
  let a = this.state.matrix
  // si il n'y a pas de matrice pas besoin de calculer
  if (this.state.oldMatrix !== null) {
    // transformation de x
    x = this.state.do === 0 ? 1 - x : x
    
    let base = this.state.oldMatrix
    let change = this.state.matrix
    let isNormal = true

    if (base.length < change.length) {
      [base, change] = [change, base]
      isNormal = false
    }

    a = base.map((row, ri) =>
      row.map((cell, ci) => {
        let cellChange = 0
        let cellBase = cell
        if (change[ri] && change[ri][ci])
          cellChange = change[ri][ci]
        if (!isNormal) {
          cellBase = cellChange
          cellChange = cell
        }
        return Math.round(cellBase + x * (cellChange - cellBase))
      })
    )
  }
  // quand c'est termine, force la matrice
  if (x === 1) {
    a = this.state.matrix
  }
\end{lstlisting}
\pagebreak
\subsubsection{Rendu du diagramme}
Une fois que la matrice est calculée, il suffit de retourner le composant ChordDiagram pour l'afficher.
\begin{lstlisting}[language=JavaScript]
  return (
    <ChordDiagram
      matrix={a}
      componentId={1}
      groupLabels={this.labels}
      groupColors={this.colors}
    />
  )
\end{lstlisting}
NB: les labels et couleurs n'ont pas encore été définies mais le seront dans une prochaine partie.

\subsubsection{Création des données de matrices}
Etant donné qu'il n'y a pas de vrai données, il va falloir les générer aléatoirement. Voici une fonction qui va créer une matrice. Il suffit d l'ajouter dans les méthodes de la classe App.
\begin{lstlisting}[language=JavaScript]
  generateMatrix (size = 4, min = 100, max = 10000) {
    let item = []
    for(let i=0; i<size; i++) item.push(0)
    return item.map(_=> item.map(_=>
      Math.round(Math.random()*(max-min)+min)
    ))
  }
\end{lstlisting}

\subsubsection{Méthode de màj de matrice}


\end{document}